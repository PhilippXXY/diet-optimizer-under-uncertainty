\documentclass[a4paper,12pt]{article}

\usepackage[applemac]{inputenc}
\usepackage[T1]{fontenc}
\usepackage[english]{babel}
\usepackage[nospace,noadjust]{cite}
\usepackage{eurosym}
\usepackage{amssymb,amsmath}
\usepackage{graphicx}
\usepackage{color}
\definecolor{kit}{cmyk}{1,0,0.6,0}
\usepackage{hyperref}
\hypersetup{
    pdftoolbar=true,
    pdfmenubar=true,
    pdfpagemode=UseOutlines,
    bookmarksnumbered=true,
    linktocpage=true,
    colorlinks=false,
    colorlinks=false
}

\setlength{\parindent}{0pt}
\parskip1.5ex

\begin{document}
\begin{titlepage}
    \begin{center}
        \vspace*{-80pt}
        \includegraphics[scale=0.25]{img/kit_logo.png}

        \vspace*{40pt}

        Department of Economics and Management \\[1ex]
        Institute for Operations Research (IOR) \\[1ex]
        Optimization under Uncertainty \\[1ex]
        Prof. Dr. Steffen Rebennack          

        \vspace*{25pt}   

        \vspace*{35pt}

        \fboxsep 40pt
        \fboxrule 6pt
        \fcolorbox{kit}{white}{
            \parbox{80mm}{
                \begin{center}
                    \Large{Excercise Submission}\\ 
                    \Large{Winter Semester 2024/25}
                \end{center}
            }
        }

        \vspace*{40pt}

        \normalsize

        First Name Last Name\\
        Student ID:\\
        Study Program (B.Sc.)\\[4ex]

        und \\[4ex]

        First Name Last Name\\
        Student ID:\\
        Study Program (B.Sc.)\\[4ex]
    \end{center}
\end{titlepage}

\newpage

\section*{Solutions to Task 2}

The uncertainty set can generally be described as follows:

\[
\mathcal{U} = \left\{
\underbrace{
\left[
\begin{array}{c|c}
(c^0)^\top & d^0 \\ \hline
A^0 & b^0
\end{array}
\right]
}_{D^0}
+ \sum_{\ell=1}^L \zeta_\ell
\underbrace{
\left[
\begin{array}{c|c}
(c^\ell)^\top & d^\ell \\ \hline
A^\ell & b^\ell
\end{array}
\right]
}_{D^\ell}
\;\middle|\; \zeta \in \mathbb{Z}
\right\}
\]

For the task, among other things, the objective function is given by
\[
    \min \quad 0.22x_1 + 0.18x_2 + 0.07x_3 + 0.14x_4 + 0.55x_5 + 0.1x_6 + 0.54x_7 + 0.28x_8 + 3.2x_9
\]
and the second constraint is
\[
0.35x_1 + 7x_2 + x_3 + 2x_4 + 25x_5 + 3.5x_6 + 9x_7 + 2.5x_8 + 21x_9 \geq 56
\]
required.

First, the second constraint must be converted to standard form. This is done by multiplying the inequality by $-1$:
\[
    -0.35x_1  -7x_2  -x_3  -2x_4 - 25x_5 - 3.5x_6 - 9x_7 - 2.5x_8 - 21x_9 \leq -56
\]

From the corresponding considerations, the nominal data matrix $D^0_2$ can be directly derived: 
\[
    D^0_2 = \begin{pmatrix}
        0.22 & 0.18 & 0.07 & 0.14 & 0.55 & 0.1 & 0.54 & 0.28 & 3.2 & 0 \\
        -0.35 & -7 & -1 & -2 & -25 & -3.5 & -9 & -2.5 & -21 & -56
    \end{pmatrix}
\]
        
The uncertainties affect all coefficients of the objective function, as well as the nutrient values for proteins, fats, calcium and vitamin B2. Additionally, there is a fluctuation in the minimum protein requierement of $10g$. All values can be taken from the table in the task description.

Therefore, it is crucial to calculate the uncertainties of the objective and constraint functions in absolute values. The uncertainty magnitude can then be represented as follows:
\[
    \text{OF} = \begin{pmatrix}
        0.06 & \underbrace{0.027}_{\mathrm{0.18 \times 0.15}} & \underbrace{0.014}_{\mathrm{0.07 \times 0.2}} & 0.04 & 0.1 & \underbrace{0.025}_{\mathrm{0.1 \times 0.25}} & \underbrace{0.216}_{\mathrm{0.54 \times 0.4}} & 0.1 & \underbrace{1.28}_{\mathrm{3.2 \times 0.4}}
    \end{pmatrix}        
\]
\[
    \text{CF} = \begin{pmatrix}
        \underbrace{-0.07}_{\mathrm{-0.35 \times 0.2}} & \underbrace{-0.7}_{\mathrm{-7 \times 0.1}} & \underbrace{-0.2}_{\mathrm{-1 \times 0.2}} & \underbrace{-0.1}_{\mathrm{-2 \times 0.05}} & \underbrace{-0.25}_{\mathrm{-25 \times 0.01}} & \underbrace{-0.35}_{\mathrm{-3.5 \times 0.1}} & \underbrace{-0.09}_{\mathrm{-9 \times 0.01}} & \underbrace{-0.25}_{\mathrm{-2.5 \times 0.1}} & \underbrace{-3.15}_{\mathrm{-21 \times 0.15}}
    \end{pmatrix}
\]

Based on these values, all shift matrices $D^\ell_2$ can be calculated. The amount of shift matrices corresponds to the number of uncertain coefficients, which is $L = 19$ in this case. They are calculated as follows:

\[
D^1_2 = \begin{pmatrix}
    0.06 & 0 & 0 & 0 & 0 & 0 & 0 & 0 & 0 & 0 \\
    0 & 0 & 0 & 0 & 0 & 0 & 0 & 0 & 0 & 0
\end{pmatrix}
\]
\[
D^2_2 = \begin{pmatrix}
    0 & 0.027 & 0 & 0 & 0 & 0 & 0 & 0 & 0 & 0 \\
    0 & 0 & 0 & 0 & 0 & 0 & 0 & 0 & 0 & 0
\end{pmatrix}
\]
\[
D^3_2 = \begin{pmatrix}
    0 & 0 & 0.014 & 0 & 0 & 0 & 0 & 0 & 0 & 0 \\
    0 & 0 & 0 & 0 & 0 & 0 & 0 & 0 & 0 & 0
\end{pmatrix}
\]
\[
D^4_2 = \begin{pmatrix}
    0 & 0 & 0 & 0.04 & 0 & 0 & 0 & 0 & 0 & 0 \\
    0 & 0 & 0 & 0 & 0 & 0 & 0 & 0 & 0 & 0
\end{pmatrix}
\]
\[
D^5_2 = \begin{pmatrix}
    0 & 0 & 0 & 0 & 0.1 & 0 & 0 & 0 & 0 & 0 \\
    0 & 0 & 0 & 0 & 0 & 0 & 0 & 0 & 0 & 0
\end{pmatrix}
\]
\[
D^6_2 = \begin{pmatrix}
    0 & 0 & 0 & 0 & 0 & 0.025 & 0 & 0 & 0 & 0 \\
    0 & 0 & 0 & 0 & 0 & 0 & 0 & 0 & 0 & 0
\end{pmatrix}
\]
\[
D^7_2 = \begin{pmatrix}
    0 & 0 & 0 & 0 & 0 & 0 & 0.216 & 0 & 0 & 0 \\
    0 & 0 & 0 & 0 & 0 & 0 & 0 & 0 & 0 & 0
\end{pmatrix}
\]
\[
D^8_2 = \begin{pmatrix}
    0 & 0 & 0 & 0 & 0 & 0 & 0 & 0.1 & 0 & 0 \\
    0 & 0 & 0 & 0 & 0 & 0 & 0 & 0 & 0 & 0
\end{pmatrix}
\]
\[
D^9_2 = \begin{pmatrix}
    0 & 0 & 0 & 0 & 0 & 0 & 0 & 0 & 1.28 & 0 \\
    0 & 0 & 0 & 0 & 0 & 0 & 0 & 0 & 0 & 0
\end{pmatrix}
\]
\[
D^{10}_2 = \begin{pmatrix}
    0 & 0 & 0 & 0 & 0 & 0 & 0 & 0 & 0 & 0 \\
    0.07 & 0 & 0 & 0 & 0 & 0 & 0 & 0 & 0 & 0
\end{pmatrix}
\]
\[
D^{11}_2 = \begin{pmatrix}
    0 & 0 & 0 & 0 & 0 & 0 & 0 & 0 & 0 & 0 \\
    0 & 0.7 & 0 & 0 & 0 & 0 & 0 & 0 & 0 & 0
\end{pmatrix}
\]
\[
D^{12}_2 = \begin{pmatrix}
    0 & 0 & 0 & 0 & 0 & 0 & 0 & 0 & 0 & 0 \\
    0 & 0 & 0.2 & 0 & 0 & 0 & 0 & 0 & 0 & 0
\end{pmatrix}
\]
\[
D^{13}_2 = \begin{pmatrix}
    0 & 0 & 0 & 0 & 0 & 0 & 0 & 0 & 0 & 0 \\
    0 & 0 & 0 & 0.1 & 0 & 0 & 0 & 0 & 0 & 0
\end{pmatrix}
\]
\[
D^{14}_2 = \begin{pmatrix}
    0 & 0 & 0 & 0 & 0 & 0 & 0 & 0 & 0 & 0 \\
    0 & 0 & 0 & 0 & 0.25 & 0 & 0 & 0 & 0 & 0
\end{pmatrix}
\]
\[
D^{15}_2 = \begin{pmatrix}
    0 & 0 & 0 & 0 & 0 & 0 & 0 & 0 & 0 & 0 \\
    0 & 0 & 0 & 0 & 0 & 0.35 & 0 & 0 & 0 & 0
\end{pmatrix}
\]
\[
D^{16}_2 = \begin{pmatrix}
    0 & 0 & 0 & 0 & 0 & 0 & 0 & 0 & 0 & 0 \\
    0 & 0 & 0 & 0 & 0 & 0 & 0.09 & 0 & 0 & 0
\end{pmatrix}
\]
\[
D^{17}_2 = \begin{pmatrix}
    0 & 0 & 0 & 0 & 0 & 0 & 0 & 0 & 0 & 0 \\
    0 & 0 & 0 & 0 & 0 & 0 & 0 & 0.25 & 0 & 0
\end{pmatrix}
\]
\[
D^{18}_2 = \begin{pmatrix}
    0 & 0 & 0 & 0 & 0 & 0 & 0 & 0 & 0 & 0 \\
    0 & 0 & 0 & 0 & 0 & 0 & 0 & 0 & 3.15 & 0
\end{pmatrix}
\]
\[
D^{19}_2 = \begin{pmatrix}
    0 & 0 & 0 & 0 & 0 & 0 & 0 & 0 & 0 & 0 \\
    0 & 0 & 0 & 0 & 0 & 0 & 0 & 0 & 0 & 10
\end{pmatrix}
\]

\newpage

\end{document}
