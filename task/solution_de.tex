\documentclass[a4paper,12pt]{article}

\usepackage[applemac]{inputenc}
\usepackage[T1]{fontenc}
\usepackage[ngerman]{babel}
\usepackage[nospace,noadjust]{cite}
\usepackage{eurosym}
\usepackage{amssymb,amsmath}
\usepackage{graphicx}
\usepackage{color}
\definecolor{kit}{cmyk}{1,0,0.6,0}
\usepackage{hyperref}
\hypersetup{
    pdftoolbar=true,
    pdfmenubar=true,
    pdfpagemode=UseOutlines,
    bookmarksnumbered=true,
    linktocpage=true,
    colorlinks=false,
    colorlinks=false
}

\setlength{\parindent}{0pt}
\parskip1.5ex

\begin{document}
\begin{titlepage}
    \begin{center}
        \vspace*{-80pt}
        \includegraphics[scale=0.25]{img/kit_logo.png}

        \vspace*{40pt}

        Fakult\text{\"a}t f\text{\"u}r Wirtschaftswissenschaften \\[1ex]
        Institut f\text{\"u}r Operations Research (IOR) \\[1ex]
        Optimierungsans\text{\"a}tze unter Unsicherheit \\[1ex]
        Prof. Dr. Steffen Rebennack          

        \vspace*{25pt}   

        \vspace*{35pt}

        \fboxsep 40pt
        \fboxrule 6pt
        \fcolorbox{kit}{white}{
            \parbox{80mm}{
                \begin{center}
                    \Large{Abgabge Rechner\text{\"u}bung}\\ 
                    \Large{Wintersemester 2024/25}
                \end{center}
            }
        }

        \vspace*{40pt}

        \normalsize

        Vorname Nachname\\
        Matr. Nr.:\\
        Studiengang (B.Sc.)\\[4ex]

        und \\[4ex]

        Vorname Nachname\\
        Matr. Nr.:\\
        Studiengang (B.Sc.)\\[4ex]
    \end{center}
\end{titlepage}

\newpage

\section*{L\"osungen zu Aufgabe 1}

\subsection*{Aufgabenteil a}

Der Quellcode ist in der Datei \href{../src/r1/Aufgabe1.gms}{Aufgabe1.gms} zu finden.

\subsection*{Aufgabenteil b}

Die entsprechende Ausf\"uhrung kann unter \href{../results/r1/Aufgabe1.lst}{Aufgabe1.lst} eingesehen werden. Daraus l\"asst sich entnehmen, dass der optimale Wert bei $5,169180$ Euro liegt und der optimale Punkt bei
\[
\begin{pmatrix}
\text{Apfel} \\ 
\text{Cornflakes} \\ 
\text{Karotten} \\ 
\text{Kartoffeln} \\ 
\text{K\"ase} \\ 
\text{Milch} \\ 
\text{Schokolade} \\ 
\text{Spinat} \\ 
\text{Steak}
\end{pmatrix}^T
=
\begin{pmatrix}
0 \\ 
3 \\ 
4 \\ 
0,4512 \\ 
0 \\ 
0 \\ 
2,0111 \\
0 \\
1
\end{pmatrix}^T
\]
zu finden ist.

\newpage

\end{document}
