\documentclass[a4paper,12pt]{article}

\usepackage[applemac]{inputenc}
\usepackage[T1]{fontenc}
\usepackage[ngerman]{babel}
\usepackage[nospace,noadjust]{cite}
\usepackage{eurosym}
\usepackage{amssymb,amsmath,mathdots}
\usepackage{graphicx}
\usepackage{color}
\definecolor{kit}{cmyk}{1,0,0.6,0}
\usepackage{hyperref}
\usepackage{breqn}
\usepackage{changepage}

\hypersetup{
    pdftoolbar=true,
    pdfmenubar=true,
    pdfpagemode=UseOutlines,
    bookmarksnumbered=true,
    linktocpage=true,
    colorlinks=false,
    colorlinks=false
}

\setlength{\parindent}{0pt}
\parskip1.5ex

\begin{document}
\begin{titlepage}
    \begin{center}
        \vspace*{-80pt}
        %\includegraphics[scale=0.25]{img/kit_logo.png}

        \vspace*{40pt}

        Fakult\text{\"a}t f\text{\"u}r Wirtschaftswissenschaften \\[1ex]
        Institut f\text{\"u}r Operations Research (IOR) \\[1ex]
        Optimierungsans\text{\"a}tze unter Unsicherheit \\[1ex]
        Prof. Dr. Steffen Rebennack          

        \vspace*{25pt}   

        \vspace*{35pt}

        \fboxsep 40pt
        \fboxrule 6pt
        \fcolorbox{kit}{white}{
            \parbox{80mm}{
                \begin{center}
                    \Large{Abgabge Rechner\text{\"u}bung}\\ 
                    \Large{Wintersemester 2024/25}
                \end{center}
            }
        }

        \vspace*{40pt}

        \normalsize

        Vorname Nachname\\
        Matr. Nr.:\\
        Studiengang (B.Sc.)\\[4ex]

        und \\[4ex]

        Vorname Nachname\\
        Matr. Nr.:\\
        Studiengang (B.Sc.)\\[4ex]
    \end{center}
\end{titlepage}

\newpage

\section*{L\"osungen zu Aufgabe 3}

\subsection*{Aufgabenteil a - Intervall-Unsicherheit}

\subsubsection*{Umwandeln des Ursprungsproblems in Standardform}

Das Ursprungsproblem muss in folgende Struktur umgewandelt werden:
\[
(\text{LO}) \quad \left\{ \min_{x} \left\{ \mathbf{c}^\top \mathbf{x} + d \; \middle| \; A\mathbf{x} \leq \mathbf{b} \right\} \right\}
\]
Entsprechend l\"asst sich dieses dann nun folgenderma\ss en darstellen:
\[
\begin{aligned}
    \min & \quad 0.22x_1 + 0.18x_2 + 0.07x_3 + 0.14x_4 + 0.55x_5 + 0.16x_6 + 0.54x_7 + 0.28x_8 + 3.2x_9 \\
    \text{s.t.} & \quad -52x_1 - 355x_2 - 26x_3 - 71x_4 - 354x_5 - 64x_6 - 536x_7 - 17x_8 - 121x_9 \leq -2400 \\
    & \quad -0.35x_1 - 7x_2 - x_3 - 2x_4 - 25x_5 - 3.5x_6 - 9x_7 - 2.5x_8 - 21x_9 \leq -56 \\
    & \quad -18x_1 - 307x_2 - 7x_3 - 24.5x_4 + 177x_5 + 12x_6 + 52x_7 + 6.5x_8 + 60.5x_9 \leq 0 \\
    & \quad -0.4x_1 - 0.6x_2 - 0.2x_3 - 0.11x_4 - 28.3x_5 - 3.5x_6 - 31.5x_7 - 0.3x_8 - 4x_9 \leq -50 \\
    & \quad 0.4x_1 + 0.6x_2 + 0.2x_3 + 0.11x_4 + 28.3x_5 + 3.5x_6 + 31.5x_7 + 0.3x_8 + 4x_9 \leq 70 & \\
    & \quad -7x_1 - 13x_2 - 41x_3 - 6x_4 - 800x_5 - 120x_6 - 214x_7 - 126x_8 - 3x_9 \leq -500 \\
    & \quad -30x_1 - 60x_2 - 53x_3 - 47x_4 - 300x_5 - 170x_6 - 370x_7 - 230x_8 - 130x_9 \leq -1100 \\
    & \quad -x_1 - x_3 - x_8 \leq -4 \\
    & \quad x_1, \dots, x_9 \leq 5 \\
    & \quad - x_1, \dots, - x_9 \leq 0 \\
    & \quad x_2 \leq 3 \\
    & \quad x_6 \leq 2 \\
    & \quad x_7 \leq 3 \\
    & \quad - x_9 \leq - 1
\end{aligned}
\]

\subsubsection*{Bilden der nominalen Datenmatrix $D^0$}

Mit den vorherigen \"Uberlegungen l\"asst sich ganz einfach die nominale Datenmatrix $D^0$ dieser Struktur
\[
D^0 = \begin{pmatrix}\begin{array}{c|c}
(c^0)^\top & d^0 \\ \hline
A^0 & b^0
\end{array}\end{pmatrix}
\]
bilden.

Nach dem Streichen von Zeilen ohne Unsicherheiten lautet diese dann wie folgt:
\[
D^0 = \begin{pmatrix}
0.22 & 0.18 & 0.07 & 0.14 & 0.55 & 0.16 & 0.54 & 0.28 & 3.2 & 0 \\ % Zielfunktion
-52 & -355 & -26 & -71 & -354 & -64 & -536 & -17 & -121 & -2400 \\ % Kalorien
-0.35 & -7 & -1 & -2 & -25 & -3.5 & -9 & -2.5 & -21 & -56 \\ % Mind. Proteine
-0.4 & -0.6 & -0.2 & -0.11 & -28.3 & -3.5 & -31.5 & -0.3 & -4 & -50 \\ % Mind. Fett
0.4 & 0.6 & 0.2 & 0.11 & 28.3 & 3.5 & 31.5 & 0.3 & 4 & 70 \\ % Max. Fett
-7 & -13 & -41 & -6 & -800 & -120 & -214 & -126 & -3 & -500 \\ % Mind. Calcium
-30 & -60 & -53 & -47 & -300 & -170 & -370 & -230 & -130 & -1100 \\ % Mind. Vit. B2
\end{pmatrix}
\]

\subsubsection*{3. Bilden der Shiftmatrizen $D^\ell$}
Die Anzahl der Shiftmatrizen $D^\ell$ ergibt sich aus der Anzahl der Unsicherheiten. Alle Unsicherheiten lassen sich aus den Tabellen entnehmen.

F\"ur die Shiftmatrizen $D^\ell$ gilt es, die Unsicherheiten ggf. durch Multiplikation mit dem entsprechenden Koeffizienten der Zielfunktion in absolute Abweichungen umzuwandeln.
Wie in \"{U}bung $3f$ und Vorlesung $7$ (Abschnitt 5.3) erw\"{a}hnt, nutzt man f\"{u}r jede Unsicherheit in einer Zeile eine neue Shiftmatrix.
Durch diesen Ansatz lassen sich die Unsicherheiten eines Parameters $x_{\ell \in [1,9]}$ und des Mindestbedarfs $b$ in einer Shiftmatrix $D^\ell$ darstellen. Au\ss erdem ist es zu Vereinfachungszwecken erlaubt, Zeilen, in denen keine Unsicherheiten auftreten, zu streichen.

Die Shiftmatrizen lauten dann wie folgt:

\[
D^1_{x_1} = \begin{pmatrix}
0.06 & 0 & 0 & 0 & 0 & 0 & 0 & 0 & 0 & 0 \\ % Zielfunktion
0.07 & 0 & 0 & 0 & 0 & 0 & 0 & 0 & 0 & 0 \\ % Proteine
0.8 & 0 & 0 & 0 & 0 & 0 & 0 & 0 & 0 & 0 \\ % Mind. Fette
0.8 & 0 & 0 & 0 & 0 & 0 & 0 & 0 & 0 & 0 \\ % Max. Fette
1.4 & 0 & 0 & 0 & 0 & 0 & 0 & 0 & 0 & 0 \\ % Calcium
4.5 & 0 & 0 & 0 & 0 & 0 & 0 & 0 & 0 & 0 % Vit. B2
\end{pmatrix}
\begin{pmatrix}
    Zielfunktion \\
    Proteine \\
    Min. Fette \\
    Max. Fette \\
    Calcium \\
    Vit. B2
\end{pmatrix}
\]
\[
D^2_{x_2} = \begin{pmatrix}
0 & 0.027 & 0 & 0 & 0 & 0 & 0 & 0 & 0 & 0 \\ % Zielfunktion
0 & 0.7 & 0 & 0 & 0 & 0 & 0 & 0 & 0 & 0 \\ % Proteine
0 & 0.12 & 0 & 0 & 0 & 0 & 0 & 0 & 0 & 0 \\ % Mind. Fette
0 & 0.12 & 0 & 0 & 0 & 0 & 0 & 0 & 0 & 0 \\ % Max. Fette
0 & 6.5 & 0 & 0 & 0 & 0 & 0 & 0 & 0 & 0 \\ % Calcium
0 & 12 & 0 & 0 & 0 & 0 & 0 & 0 & 0 & 0 % Vit. B2
\end{pmatrix}
\begin{pmatrix}
    Zielfunktion \\
    Proteine \\
    Min. Fette \\
    Max. Fette \\
    Calcium \\
    Vit. B2
\end{pmatrix}
\]
\[
D^3_{x_3} = \begin{pmatrix} 
0 & 0 & 0.014 & 0 & 0 & 0 & 0 & 0 & 0 & 0 \\ % Zielfunktion
0 & 0 & 0.2 & 0 & 0 & 0 & 0 & 0 & 0 & 0 \\ % Proteine
0 & 0 & 0.04 & 0 & 0 & 0 & 0 & 0 & 0 & 0 \\ % Mind. Fette
0 & 0 & 0.04 & 0 & 0 & 0 & 0 & 0 & 0 & 0 \\ % Max. Fette
0 & 0 & 8.2 & 0 & 0 & 0 & 0 & 0 & 0 & 0 \\ % Calcium
0 & 0 & 10.6 & 0 & 0 & 0 & 0 & 0 & 0 & 0
\end{pmatrix}
\begin{pmatrix}
    Zielfunktion \\
    Proteine \\
    Min. Fette \\
    Max. Fette \\
    Calcium \\
    Vit. B2
\end{pmatrix}
\]
\[
D^4_{x_4} = \begin{pmatrix}
0 & 0 & 0 & 0.04 & 0 & 0 & 0 & 0 & 0 & 0 \\ % Zielfunktion
0 & 0 & 0 & 0.1 & 0 & 0 & 0 & 0 & 0 & 0 \\ % Proteine
0 & 0 & 0 & 0.0022 & 0 & 0 & 0 & 0 & 0 & 0 \\ % Mind. Fette
0 & 0 & 0 & 0.0022 & 0 & 0 & 0 & 0 & 0 & 0 \\ % Max. Fette
0 & 0 & 0 & 0.12 & 0 & 0 & 0 & 0 & 0 & 0 \\ % Calcium
0 & 0 & 0 & 0.47 & 0 & 0 & 0 & 0 & 0 & 0 % Vit. B2
\end{pmatrix}
\begin{pmatrix}
    Zielfunktion \\
    Proteine \\
    Min. Fette \\
    Max. Fette \\
    Calcium \\
    Vit. B2
\end{pmatrix}
\]
\[
D^5_{x_5} = \begin{pmatrix}
0 & 0 & 0 & 0 & 0.1 & 0 & 0 & 0 & 0 & 0 \\ % Zielfunktion
0 & 0 & 0 & 0 & 1.77 & 0 & 0 & 0 & 0 & 0 \\ % Proteine
0 & 0 & 0 & 0 & 0.035 & 0 & 0 & 0 & 0 & 0 \\ % Mind. Fette
0 & 0 & 0 & 0 & 0.035 & 0 & 0 & 0 & 0 & 0 \\ % Max. Fette
0 & 0 & 0 & 0 & 80 & 0 & 0 & 0 & 0 & 0 \\ % Calcium
0 & 0 & 0 & 0 & 15 & 0 & 0 & 0 & 0 & 0 % Vit. B2
\end{pmatrix}
\begin{pmatrix}
    Zielfunktion \\
    Proteine \\
    Min. Fette \\
    Max. Fette \\
    Calcium \\
    Vit. B2
\end{pmatrix}
\]
\[
D^6_{x_6} = \begin{pmatrix}
0 & 0 & 0 & 0 & 0 & 0.04 & 0 & 0 & 0 & 0 \\ % Zielfunktion
0 & 0 & 0 & 0 & 0 & 0.35 & 0 & 0 & 0 & 0 \\ % Proteine
0 & 0 & 0 & 0 & 0 & 0.35 & 0 & 0 & 0 & 0 \\ % Mind. Fette
0 & 0 & 0 & 0 & 0 & 0.35 & 0 & 0 & 0 & 0 \\ % Max. Fette
0 & 0 & 0 & 0 & 0 & 24 & 0 & 0 & 0 & 0 \\ % Calcium
0 & 0 & 0 & 0 & 0 & 8.5 & 0 & 0 & 0 & 0 % Vit. B2
\end{pmatrix}
\begin{pmatrix}
    Zielfunktion \\
    Proteine \\
    Min. Fette \\
    Max. Fette \\
    Calcium \\
    Vit. B2
\end{pmatrix}
\]
\[
D^7_{x_7} = \begin{pmatrix}
0 & 0 & 0 & 0 & 0 & 0 & 0.216 & 0 & 0 & 0 \\ % Zielfunktion
0 & 0 & 0 & 0 & 0 & 0 & 0.9 & 0 & 0 & 0 \\ % Proteine
0 & 0 & 0 & 0 & 0 & 0 & 3.15 & 0 & 0 & 0 \\ % Mind. Fette
0 & 0 & 0 & 0 & 0 & 0 & 3.15 & 0 & 0 & 0 \\ % Max. Fette
0 & 0 & 0 & 0 & 0 & 0 & 21.4 & 0 & 0 & 0 \\ % Calcium
0 & 0 & 0 & 0 & 0 & 0 & 37 & 0 & 0 & 0 % Vit. B2
\end{pmatrix}
\begin{pmatrix}
    Zielfunktion \\
    Proteine \\
    Min. Fette \\
    Max. Fette \\
    Calcium \\
    Vit. B2
\end{pmatrix}
\]
\[
D^8_{x_8} = \begin{pmatrix}
0 & 0 & 0 & 0 & 0 & 0 & 0 & 0.1 & 0 & 0 \\ % Zielfunktion
0 & 0 & 0 & 0 & 0 & 0 & 0 & 0.25 & 0 & 0 \\ % Proteine
0 & 0 & 0 & 0 & 0 & 0 & 0 & 0.045 & 0 & 0 \\ % Mind. Fette
0 & 0 & 0 & 0 & 0 & 0 & 0 & 0.045 & 0 & 0 \\ % Max. Fette
0 & 0 & 0 & 0 & 0 & 0 & 0 & 12.6 & 0 & 0 \\ % Calcium
0 & 0 & 0 & 0 & 0 & 0 & 0 & 46 & 0 & 0 % Vit. B2
\end{pmatrix}
\begin{pmatrix}
    Zielfunktion \\
    Proteine \\
    Min. Fette \\
    Max. Fette \\
    Calcium \\
    Vit. B2
\end{pmatrix}
\]
\[
D^9_{x_9} = \begin{pmatrix}
0 & 0 & 0 & 0 & 0 & 0 & 0 & 0 & 1.28 & 0 \\ % Zielfunktion
0 & 0 & 0 & 0 & 0 & 0 & 0 & 0 & 3.15 & 0 \\ % Proteine
0 & 0 & 0 & 0 & 0 & 0 & 0 & 0 & 1.2 & 0 \\ % Mind. Fette
0 & 0 & 0 & 0 & 0 & 0 & 0 & 0 & 1.2 & 0 \\ % Max. Fette
0 & 0 & 0 & 0 & 0 & 0 & 0 & 0 & 0.6 & 0 \\ % Calcium
0 & 0 & 0 & 0 & 0 & 0 & 0 & 0 & 19.5 & 0 % Vit. B2
\end{pmatrix}
\begin{pmatrix}
    Zielfunktion \\
    Proteine \\
    Min. Fette \\
    Max. Fette \\
    Calcium \\
    Vit. B2
\end{pmatrix}
\]
\[
D^{10}_{b} = \begin{pmatrix}
0 & 0 & 0 & 0 & 0 & 0 & 0 & 0 & 0 & 350 \\ % b_1
0 & 0 & 0 & 0 & 0 & 0 & 0 & 0 & 0 & 10 \\ % b_2
0 & 0 & 0 & 0 & 0 & 0 & 0 & 0 & 0 & 10 \\ % b_4
0 & 0 & 0 & 0 & 0 & 0 & 0 & 0 & 0 & 200 \\ % b_6
0 & 0 & 0 & 0 & 0 & 0 & 0 & 0 & 0 & 300 % b_7
\end{pmatrix}
\begin{pmatrix}
    Kalorien \\
    Proteine \\
    Min. Fette \\
    Calcium \\
    Vit. B2
\end{pmatrix}
\]

Insgesamt gilt dann:

\[
\mathcal{U} = \left\{ D^0 + \sum_{\ell=1}^{10}D^{\ell} \zeta_{\ell} \;\vert\; \zeta_{\ell} \in [-1, 1] \right\}
\]

\subsubsection*{Bilden des robusten Pendants}

Mit Hilfe der Unsicherheitsmenge aus dem vorherigen Schritt l\"asst sich nun das robuste Pendant des Ursprungsproblems bilden:
\[
\begin{aligned}
    &\min && \sup_{(c, d, \cdot) \in \mathcal{U}} \left(c^\top x + d\right) \\
    &\text{s.t.} && Ax \leq b \quad \forall (\cdot, A, b) \in \mathcal{U}, \\
    & && x \geq 0
\end{aligned}
\]
In diesem Fall ergibt es sich zu:
\[
\begin{aligned}
   \begin{aligned}
    \min & \quad \textcolor{blue}{c^\top x \quad \forall (c^\top, \cdot,\cdot,\cdot) \in \mathcal{U}} \\ % TODO
    \text{s.t.} & \quad \textcolor{blue}{a_1^\top x \leq b_1 \quad \forall (\cdot, \cdot, a_1^\top, b_1) \in \mathcal{U}} \\
    & \quad \textcolor{blue}{a_2^\top x \leq b_2 \quad \forall (\cdot, \cdot, a_2^\top, b_2) \in \mathcal{U}} \\
    & \quad -18x_1 - 307x_2 - 7x_3 - 24.5x_4 + 177x_5 + 12x_6 + 52x_7 + 6.5x_8 + 60.5x_9 \leq 0 \\
    & \quad \textcolor{blue}{a_4^\top x \leq b_4 \quad \forall (\cdot, \cdot, a_4^\top, b_4) \in \mathcal{U}} \\
    & \quad \textcolor{blue}{a_5^\top x \leq b_5 \quad \forall (\cdot, \cdot, a_5^\top, b_5) \in \mathcal{U}} \\
    & \quad \textcolor{blue}{a_6^\top x \leq b_6 \quad \forall (\cdot, \cdot, a_6^\top, b_6) \in \mathcal{U}} \\
    & \quad \textcolor{blue}{a_7^\top x \leq b_7 \quad \forall (\cdot, \cdot, a_7^\top, b_7) \in \mathcal{U}}\\
    & \quad -x_1 - x_3 - x_8 \leq -4 \\
    & \quad x_1, \dots, x_9 \leq 5 \\
    & \quad - x_1, \dots, - x_9 \leq 0 \\
    & \quad x_2 \leq 3 \\
    & \quad x_6 \leq 2 \\
    & \quad x_7 \leq 3 \\
    & \quad - x_9 \leq - 1
\end{aligned}
\end{aligned}
\]

\subsubsection*{Herleitung eines LP-\"Aquivalents}

Zur Veranschaulichung wird die zweite Nebenbedingung (Proteine) umgewandelt.
Dabei wird die Definition von $\mathcal{U}$ in die Nebenbedingung eingesetzt:
\[
(a_1^0)^\top x + \zeta_1 (a_1^1)^\top x + \dots + \zeta_9 (a_1^9)^\top x + \underbrace{\zeta_{10} (a_1^{10})^\top x}_{= 0} \leq b_1^0 + \underbrace{\zeta_1 b_1^1}_{=0} + \dots +\zeta_{10} b_1^{10} \quad \forall \zeta \in [-1, 1]
\]

Zuerst gilt es die Projektion der Unsicherheitsmenge bei Intervall-Unsicherheit zu bestimmen:
\[
\mathcal{U}^{2} = \left\{ 
\begin{array}{l}
    \begin{pmatrix}
    -0.35 \\ 
    -7 \\ 
    -1 \\
    -2 \\
    -25 \\
    -3.5 \\
    -9 \\
    -2.5 \\
    -21 \\
    -56
    \end{pmatrix}
    + \zeta_1 
    \begin{pmatrix}
    0.07 \\ 
    0 \\ 
    0 \\
    0 \\
    0 \\
    0 \\
    0 \\
    0 \\
    0 \\
    0
    \end{pmatrix}
    + \zeta_2 
    \begin{pmatrix}
    0 \\ 
    0.7 \\ 
    0 \\
    0 \\
    0 \\
    0 \\
    0 \\
    0 \\
    0 \\
    0
    \end{pmatrix}
    + \zeta_3 
    \begin{pmatrix}
    0 \\ 
    0 \\ 
    0.2 \\
    0 \\ 
    0 \\ 
    0 \\ 
    0 \\ 
    0 \\ 
    0 \\
    0
    \end{pmatrix}
    + \zeta_4 
    \begin{pmatrix}
    0 \\ 
    0 \\ 
    0 \\
    0.1 \\ 
    0 \\ 
    0 \\ 
    0 \\ 
    0 \\ 
    0 \\
    0
    \end{pmatrix}
    + \zeta_5 
    \begin{pmatrix}
    0 \\ 
    0 \\ 
    0\\
    0 \\ 
    1.77 \\ 
    0 \\ 
    0 \\ 
    0 \\ 
    0 \\
    0
    \end{pmatrix}
    \\[10pt]
    + \zeta_6 
    \begin{pmatrix}
    0 \\ 
    0 \\ 
    0 \\
    0 \\ 
    0 \\ 
    0.35 \\ 
    0 \\ 
    0 \\ 
    0 \\
    0
    \end{pmatrix}
    + \zeta_7 
    \begin{pmatrix}
    0 \\ 
    0 \\ 
    0 \\
    0 \\ 
    0 \\ 
    0 \\ 
    0.9 \\ 
    0 \\ 
    0 \\
    0
    \end{pmatrix}
    + \zeta_8 
    \begin{pmatrix}
    0 \\ 
    0 \\ 
    0 \\
    0 \\ 
    0 \\ 
    0 \\ 
    0 \\ 
    0.25 \\ 
    0 \\
    0
    \end{pmatrix}
    + \zeta_9
    \begin{pmatrix}
    0 \\ 
    0 \\ 
    0 \\
    0 \\ 
    0 \\ 
    0 \\ 
    0 \\ 
    0 \\ 
    3.15 \\
    0
    \end{pmatrix}
    + \zeta_{10} 
    \begin{pmatrix}
    0 \\ 
    0 \\ 
    0 \\
    0 \\ 
    0 \\ 
    0 \\ 
    0 \\ 
    0 \\ 
    0 \\
    10
    \end{pmatrix}
\end{array}
\; \middle| \; \|\zeta\|_\infty \leq 1 
\right\}.
\]

Damit ergibt sich f\"ur die Restriktion folgende robuste Darstellung:
\[
\begin{aligned}
    &\quad-0.35x_1 - 7x_2 - x_3 - 2x_4 - 25x_5 - 3.5x_6 - 9x_7 - 2.5x_8 - 21x_9 \\
    &\qquad + \max_{\|\zeta\|_\infty \leq 1} \big( 0.07\zeta_1 + 0.7\zeta_2 + 0.2\zeta_3 + 0.1\zeta_4 + 1.77\zeta_5 +  0.35\zeta_6\\
    &\qquad + 0.9\zeta_7 + 0.25\zeta_8 + 3.15\zeta_9 - 10\zeta_{10} \big) \leq -56
\end{aligned}
\]
\[
\begin{aligned}
    &\Leftrightarrow -0.35x_1 - 7x_2 - x_3 - 2x_4 - 25x_5 - 3.5x_6 - 9x_7 - 2.5x_8 - 21x_9 \\
    &\qquad + 0.07|\zeta_1| + 0.7|\zeta_2| + 0.2|\zeta_3| + 0.1|\zeta_4| + 1.77|\zeta_5| + 0.35|\zeta_6| \\
    &\qquad + 0.9|\zeta_7| + 0.25|\zeta_8| + 3.15|\zeta_9| + 10|\zeta_{10}| \leq -56
\end{aligned}
\]
\newpage
Ein lineares \"Aquivalent l\"asst sich daraufhin mit Hilfe des Lifting-Ansatzes aufstellen:
\[
\begin{aligned}
    &-0.35x_1 - 7x_2 - x_3 - 2x_4 - 25x_5 - 3.5x_6 - 9x_7 - 2.5x_8 - 21x_9 \\
    &\quad + w_{21} + w_{22} + w_{23} + w_{24} + w_{25} + w_{26} + w_{27}+ w_{28}+ w_{29}+ w_{210}
\end{aligned}
\begin{aligned}
    &\quad \leq -56
\end{aligned}
\]
\[
\begin{aligned} 
    -w_{21} \leq 0.07x_1 \leq w_{21} \\
    -w_{22} \leq 0.7x_2 \leq w_{22} \\
    -w_{23} \leq 0.2x_3 \leq w_{23} \\
    -w_{24} \leq 0.1x_4 \leq w_{24} \\
    -w_{25} \leq 1.77x_5 \leq w_{25} \\
    -w_{26} \leq 0.35x_6 \leq w_{26} \\
    -w_{27} \leq 0.9x_7 \leq w_{27} \\
    -w_{28} \leq 0.25x_8 \leq w_{28} \\
    -w_{29} \leq 3.15x_9 \leq w_{29} \\
    -w_{210} \leq 10 \leq w_{210} \\
\end{aligned}
\]

Die restlichen Schritte zur Herleitung des linearen \"Aquivalents wurden analog durchgef\"uhrt. Man findet das gesamte lineare Programm unter dem Abschnitt \hyperref[sec:lp-equivalent-interval]{LP-\"Aquivalent bei Intervall-Unsicherheit}.

\newpage

\subsubsection*{L\"osung des linearen Programms}

Nach der Herleitung des linearen \"Aquivalents kann das lineare Programm gel\"ost werden. Hierzu wurde der Code entsprechend in \href{../src/r3/Aufgabe3a.gms}{Aufgabe3a.gms} umgesetzt.
Daraus l\"asst sich entnehmen, dass die optimale L\"osung in folgendem Punkt liegt:
\[
x^* = \begin{pmatrix}
    \text{Apfel} \\ \text{Cornflakes} \\ \text{Karotten} \\ \text{Kartoffeln} \\ \text{K\"ase} \\ \text{Milch} \\ \text{Schokolade} \\ \text{Spinat} \\ \text{Steak}
    \end{pmatrix}^T
     =
     \begin{pmatrix}
     0, 3, 4, 4.7749, 0, 1.0713, 1.9635, 0, 1
        \end{pmatrix}
\]
Der optimale Zielfunktionswert betr\"agt \EUR{7.914778}.

\subsubsection*{Zusammensetzung der Variablen und Restriktionen} 

Die Variablen und Restriktionen des linearen Programms besteht aus $9$ Entscheidungsvariablen $x_i$, $51$ Unsicherheitsvariablen in den Nebenbedingungen $w_{ai}$ und $9$ Unsicherheitsvariablen in der Zielfunktion $w_{0i}$.
Zu den vorher bestehenden Restriktionen kommen je Unsicherheitsvariable eine Restriktion hinzu, die die Unsicherheiten beschreiben, d.h. es kommen auf die bestehenden $14$ Restriktionen $60$ hinzu.
Demnach besteht das lineare Programm aus $69$ Variablen und $74$ Restriktionen.

\newpage

\subsection*{Aufgabenteil b - Ellips-Unsicherheit}

\subsubsection*{Herleitung eines LP-\"Aquivalents}

Um das lineare \"Aquivalent f\"ur Ellips-Unsicherheit abzuleiten, k\"onnen wir \"ahnliche Vor\"uberlegungen wie bei Intervall-Unsicherheit anwenden. Allerdings wird die Unsicherheitsmenge $\mathcal{U}$ unter Verwendung der euklidischen Norm anstelle der Supremumsnorm beschrieben. In diesem Fall lautet sie dann wie folgt:
\[
\mathcal{U} = \left\{ D^0 + \sum_{\ell=1}^{10}D^{\ell} \zeta_{\ell} \;\vert\; \|\zeta\|_2 \leq 1 \right\}
\]
Wir betrachten dieselbe Nebenbedingung wie im Fall der Intervall-Unsicherheit und erhalten das folgende robuste Pendant:
\[
\begin{aligned}
    &\quad -0.35x_1 - 7x_2 - x_3 - 2x_4 - 25x_5 - 3.5x_6 - 9x_7 - 2.5x_8 - 21x_9 \\
    &\quad + \max_{\|\zeta\|_2 \leq 1} \big( 0.07\zeta_1 + 0.7\zeta_2 + 0.2\zeta_3 + 0.1\zeta_4 + 1.77\zeta_5 +  0.35\zeta_6\\
    &\qquad + 0.9\zeta_7 + 0.25\zeta_8 + 3.15\zeta_9 - 10\zeta_{10} \big) \leq -56
\end{aligned}
\]
Da wir es mit der euklidischen Norm zu tun haben, kann die Nebenbedingung wie folgt umformuliert werden:
\[
\begin{aligned}
    &\quad -0.35x_1 - 7x_2 - x_3 - 2x_4 - 25x_5 - 3.5x_6 - 9x_7 - 2.5x_8 - 21x_9 \\
    &\quad + \sqrt{0.0049x_1^2 + 0.49x_2 + 0.04x_3^2 + 0.01x_4^2 + 3.1329x_5^2 + 0.1225x_6^2 + 0.81x_7^2 + 0.0625x_8^2 + 9.9225x_9^2 + 100} \leq -56
\end{aligned}
\]

Die restlichen Schritte zur Herleitung des linearen \"Aquivalents wurden analog durchgef\"uhrt. Man findet das gesamte lineare Programm unter dem Abschnitt \hyperref[sec:lp-equivalent-ellips]{LP-\"Aquivalent bei Ellips-Unsicherheit}.

\subsubsection*{L\"osung des Programms}

Nach der Herleitung des \"Aquivalents kann das Programm gel\"ost werden. Hierzu wurde der Code entsprechend in \href{../src/r3/Aufgabe3b.gms}{Aufgabe3b.gms} umgesetzt. Da es sich um ein nichtlineares Problem handelt, wird es mit einem entsprechenden Solver gel\"ost.
Aus der Durchf\"uhrung l\"asst sich entnehmen, dass die optimale L\"osung in folgendem Punkt liegt:
\[
x^* = \begin{pmatrix}
    \text{Apfel} \\ \text{Cornflakes} \\ \text{Karotten} \\ \text{Kartoffeln} \\ \text{K\"ase} \\ \text{Milch} \\ \text{Schokolade} \\ \text{Spinat} \\ \text{Steak}
    \end{pmatrix}^T
     =
     \begin{pmatrix}
     2.484, 3, 5, 5, 0, 0.371, 1.728, 0, 1
        \end{pmatrix}
\]
Der optimale Zielfunktionswert betr\"agt \EUR{7.6895}.

\subsubsection*{LP-\"Aquivalent bei Intervall-Unsicherheit}
\label{sec:lp-equivalent-interval}
\begin{adjustwidth}{-1.5cm}{}
\begin{tiny}
\setlength{\jot}{0pt}
\[
\begin{aligned}
   \begin{aligned}
    \min & \quad  0.22x_1 + 0.18x_2 + 0.07x_3 + 0.14x_4 + 0.55x_5 + 0.16x_6 + 0.54x_7 + 0.28x_8 + 3.2x_9 + w_{01} + w_{02} + w_{03} + w_{04} + w_{05} + w_{06} + w_{07}+ w_{08}+ w_{09} \\
    \text{s.t.} & \quad -w_{01} \leq 0.06x_1 \leq w_{01} \\
    & \quad -w_{02} \leq 0.027x_2 \leq w_{02} \\
    & \quad -w_{03} \leq 0.014x_3 \leq w_{03} \\
    & \quad -w_{04} \leq 0.04x_4 \leq w_{04} \\
    & \quad -w_{05} \leq 0.1x_5 \leq w_{05} \\
    & \quad -w_{06} \leq 0.04x_6 \leq w_{06} \\
    & \quad -w_{07} \leq 0.216x_7 \leq w_{07} \\
    & \quad -w_{08} \leq 0.1x_8 \leq w_{08} \\
    & \quad -w_{09} \leq 1.28x_9 \leq w_{09} \\
    & \quad -52x_1-355x_2-26x_3-71x_4-354x_5-64x_6-536x_7-17x_8-121x_9 \leq -2400 - w_{110}\\
    & \quad -w_{110} \leq 350 \leq w_{110} \\
    & \quad -0.35x_1 - 7x_2 - x_3 - 2x_4 - 25x_5 - 3.5x_6 - 9x_7 - 2.5x_8 - 21x_9 + w_{21} + w_{22} + w_{23} + w_{24} + w_{25} + w_{26} + w_{27}+ w_{28}+ w_{29} \leq -56 - w_{210} \\
    & \quad -w_{21} \leq 0.07x_1 \leq w_{21} \\
    & \quad -w_{22} \leq 0.7x_2 \leq w_{22} \\
    & \quad -w_{23} \leq 0.2x_3 \leq w_{23} \\
    & \quad -w_{24} \leq 0.1x_4 \leq w_{24} \\
    & \quad -w_{25} \leq 1.77x_5 \leq w_{25} \\
    & \quad -w_{26} \leq 0.35x_6 \leq w_{26} \\
    & \quad -w_{27} \leq 0.9x_7 \leq w_{27} \\
    & \quad -w_{28} \leq 0.25x_8 \leq w_{28} \\
    & \quad -w_{29} \leq 3.15x_9 \leq w_{29} \\
    & \quad -w_{210} \leq 10 \leq w_{210} \\
    & \quad -18x_1 - 307x_2 - 7x_3 - 24.5x_4 + 177x_5 + 12x_6 + 52x_7 + 6.5x_8 + 60.5x_9 \leq 0 \\
    & \quad -0.4x_1 - 0.6x_2 - 0.2x_3 - 0.11x_4 - 28.3x_5 - 3.5x_6 - 31.5x_7 - 0.3x_8 - 4x_9  + w_{41} + w_{42} + w_{43} + w_{44} + w_{45} + w_{46} + w_{47}+ w_{48}+ w_{49} \leq -50 - w_{410} \\
    & \quad -w_{41} \leq 0.8x_1 \leq w_{41} \\
    & \quad -w_{42} \leq 0.12x_2 \leq w_{42} \\
    & \quad -w_{43} \leq 0.04x_3 \leq w_{43} \\
    & \quad -w_{44} \leq 0.0022x_4 \leq w_{44} \\
    & \quad -w_{45} \leq 0.035x_5 \leq w_{45} \\
    & \quad -w_{46} \leq 0.35x_6 \leq w_{46} \\
    & \quad -w_{47} \leq 3.15x_7 \leq w_{47} \\
    & \quad -w_{48} \leq 0.045x_8 \leq w_{48} \\
    & \quad -w_{49} \leq 1.2x_9 \leq w_{49} \\
    & \quad -w_{410} \leq 10 \leq w_{410} \\
    & \quad 0.4x_1 + 0.6x_2 + 0.2x_3 + 0.11x_4 + 28.3x_5 + 3.5x_6 + 31.5x_7 + 0.3x_8 + 4x_9 + w_{51} + w_{52} + w_{53} + w_{54} + w_{55} + w_{56} + w_{57}+ w_{58}+ w_{59} \leq 70 \\
    & \quad -w_{51} \leq 0.8x_1 \leq w_{51} \\
    & \quad -w_{52} \leq 0.12x_2 \leq w_{52} \\
    & \quad -w_{53} \leq 0.04x_3 \leq w_{53} \\
    & \quad -w_{54} \leq 0.0022x_4 \leq w_{54} \\
    & \quad -w_{55} \leq 0.035x_5 \leq w_{55} \\
    & \quad -w_{56} \leq 0.35x_6 \leq w_{56} \\
    & \quad -w_{57} \leq 3.15x_7 \leq w_{57} \\
    & \quad -w_{58} \leq 0.045x_8 \leq w_{58} \\
    & \quad -w_{59} \leq 1.2x_9 \leq w_{59} \\
    & \quad -7x_1 - 13x_2 - 41x_3 - 6x_4 - 800x_5 - 120x_6 - 214x_7 - 126x_8 - 3x_9 + w_{61} + w_{62} + w_{63} + w_{64} + w_{65} + w_{66} + w_{67}+ w_{68}+ w_{69} \leq -500 - w_{610} \\
    & \quad -w_{61} \leq 1.4x_1 \leq w_{61} \\
    & \quad -w_{62} \leq 6.5x_2 \leq w_{62} \\
    & \quad -w_{63} \leq 8.2x_5 \leq w_{63} \\
    & \quad -w_{64} \leq 0.12x_4 \leq w_{64} \\
    & \quad -w_{65} \leq 80x_5 \leq w_{65} \\
    & \quad -w_{66} \leq 24x_6 \leq w_{66} \\
    & \quad -w_{67} \leq 21.4x_7 \leq w_{67} \\
    & \quad -w_{68} \leq 12.6x_8 \leq w_{68} \\
    & \quad -w_{69} \leq 0.6x_9 \leq w_{69} \\
    & \quad -w_{610} \leq 200 \leq w_{610} \\
    & \quad -30x_1 - 60x_2 - 53x_3 - 47x_4 - 300x_5 -170x_6 - 370x_7 -230x_8 - 130x_9 + w_{71} + w_{72} + w_{73} + w_{74} + w_{75} + w_{76} + w_{77}+ w_{78}+ w_{79} \leq -1100 - w_{710} \\
    & \quad -w_{71} \leq 4.5x_1 \leq w_{71} \\
    & \quad -w_{72} \leq 12x_2 \leq w_{72} \\
    & \quad -w_{73} \leq 10.6x_5 \leq w_{73} \\
    & \quad -w_{74} \leq 0.47x_4 \leq w_{74} \\
    & \quad -w_{75} \leq 15x_5 \leq w_{75} \\
    & \quad -w_{76} \leq 8.5x_6 \leq w_{76} \\
    & \quad -w_{77} \leq 37x_7 \leq w_{77} \\
    & \quad -w_{78} \leq 46x_8 \leq w_{78} \\
    & \quad -w_{79} \leq 19.5x_9 \leq w_{79} \\
    & \quad -w_{710} \leq 300 \leq w_{710} \\
    & \quad -x_1 - x_3 - x_8 \leq -4 \\
    & \quad x_1, \dots, x_9 \leq 5 \\
    & \quad - x_1, \dots, - x_9 \leq 0 \\
    & \quad x_2 \leq 3 \\
    & \quad x_6 \leq 2 \\
    & \quad x_7 \leq 3 \\
    & \quad - x_9 \leq - 1
    \end{aligned}
\end{aligned}
\]
\end{tiny}
\end{adjustwidth}

\newpage

\subsubsection*{LP-\"Aquivalent bei Ellips-Unsicherheit}
\label{sec:lp-equivalent-ellips}
\begin{adjustwidth}{-1.5cm}{}
\begin{tiny}
\setlength{\jot}{0pt}
\[
\begin{aligned}
   \begin{aligned}
    \min & \quad  0.22x_1 + 0.18x_2 + 0.07x_3 + 0.14x_4 + 0.55x_5 + 0.16x_6 + 0.54x_7 + 0.28x_8 + 3.2x_9 \\
    & \qquad+ \sqrt{0.0036x_1^2+0.000729x_2^2+0.000196x_3^2+0.0016x_4^2+0.01x_5^2+0.0016x_6^2+0.046656x_7^2+0.01x_8^2+1.6384x_9^2} \\
    \text{s.t.}
    & \quad -52x_1-355x_2-26x_3-71x_4-354x_5-64x_6-536x_7-17x_8-121x_9+\sqrt{122500} \leq -2400 \\
    & \quad -0.35x_1 - 7x_2 - x_3 - 2x_4 - 25x_5 - 3.5x_6 - 9x_7 - 2.5x_8 - 21x_9 \\
    & \qquad + \sqrt{0.0049x_1^2 + 0.49x_2^2 + 0.04x_3^2 + 0.01x_4^2 + 3.1329x_5^2 + 0.1225x_6^2 + 0.81x_7^2 + 0.0625x_8^2 + 9.9225x_9^2 + 100} \leq -56 \\
    & \quad -18x_1 - 307x_2 - 7x_3 - 24.5x_4 + 177x_5 + 12x_6 + 52x_7 + 6.5x_8 + 60.5x_9 \leq 0 \\
    & \quad -0.4x_1 - 0.6x_2 - 0.2x_3 - 0.11x_4 - 28.3x_5 - 3.5x_6 - 31.5x_7 - 0.3x_8 - 4x_9 \\
    & \qquad + \sqrt{0.64x_1^2+ 0.0144x_2^2 + 0.0016x_3^2 + 0.00000484x_4^2 + 0.001225x_5^2 + 0.1225x_6^2 + 9.9225x_7^2 + 0.002025x_8^2 + 1.44x_9^2 + 100} \leq -50 \\
    & \quad 0.4x_1 + 0.6x_2 + 0.2x_3 + 0.11x_4 + 28.3x_5 + 3.5x_6 + 31.5x_7 + 0.3x_8 + 4x_9 \\
    & \qquad + \sqrt{0.64x_1^2+ 0.0144x_2^2 + 0.0016x_3^2 + 0.00000484x_4^2 + 0.001225x_5^2 + 0.1225x_6^2 + 9.9225x_7^2 + 0.002025x_8^2 + 1.44x_9^2} \leq 70 \\
    & \quad -7x_1 - 13x_2 - 41x_3 - 6x_4 - 800x_5 - 120x_6 - 214x_7 - 126x_8 - 3x_9 \\
    & \qquad + \sqrt{1.96x_1^2 + 42.25x_2^2 + 67.24x_3^2 + 0.0144x_4^2 + 6400x_5^2 + 576x_6^2 + 457.96x_7^2 + 158.76x_8^2 + 0.36x_9^2 + 40000} \leq -500 \\
    & \quad -30x_1 - 60x_2 - 53x_3 - 47x_4 - 300x_5 -170x_6 - 370x_7 -230x_8 - 130x_9 \\
    & \qquad + \sqrt{20.25x_1^2 + 144x_2^2 + 112.36x_5^2 + 0.2209x_4^2 + 225x_5^2 + 72.25x_6^2 + 1369x_7^2 + 2116x_8^2 + 380.25x_9^2 + 90000} \leq -1100 \\
    & \quad -x_1 - x_3 - x_8 \leq -4 \\
    & \quad x_1, \dots, x_9 \leq 5 \\
    & \quad - x_1, \dots, - x_9 \leq 0 \\
    & \quad x_2 \leq 3 \\
    & \quad x_6 \leq 2 \\
    & \quad x_7 \leq 3 \\
    & \quad - x_9 \leq - 1
    \end{aligned}
\end{aligned}
\]
\end{tiny}
\end{adjustwidth}

\end{document}   