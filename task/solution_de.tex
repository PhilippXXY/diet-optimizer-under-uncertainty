\documentclass[a4paper,12pt]{article}

\usepackage[applemac]{inputenc}
\usepackage[T1]{fontenc}
\usepackage[ngerman]{babel}
\usepackage[nospace,noadjust]{cite}
\usepackage{eurosym}
\usepackage{amssymb,amsmath}
\usepackage{graphicx}
\usepackage{color}
\definecolor{kit}{cmyk}{1,0,0.6,0}
\usepackage{hyperref}
\hypersetup{
    pdftoolbar=true,
    pdfmenubar=true,
    pdfpagemode=UseOutlines,
    bookmarksnumbered=true,
    linktocpage=true,
    colorlinks=false,
    colorlinks=false
}

\setlength{\parindent}{0pt}
\parskip1.5ex

\begin{document}
\begin{titlepage}
    \begin{center}
        \vspace*{-80pt}
        \includegraphics[scale=0.25]{img/kit_logo.png}

        \vspace*{40pt}

        Fakult\text{\"a}t f\text{\"u}r Wirtschaftswissenschaften \\[1ex]
        Institut f\text{\"u}r Operations Research (IOR) \\[1ex]
        Optimierungsans\text{\"a}tze unter Unsicherheit \\[1ex]
        Prof. Dr. Steffen Rebennack          

        \vspace*{25pt}   

        \vspace*{35pt}

        \fboxsep 40pt
        \fboxrule 6pt
        \fcolorbox{kit}{white}{
            \parbox{80mm}{
                \begin{center}
                    \Large{Abgabge Rechner\text{\"u}bung}\\ 
                    \Large{Wintersemester 2024/25}
                \end{center}
            }
        }

        \vspace*{40pt}

        \normalsize

        Vorname Nachname\\
        Matr. Nr.:\\
        Studiengang (B.Sc.)\\[4ex]

        und \\[4ex]

        Vorname Nachname\\
        Matr. Nr.:\\
        Studiengang (B.Sc.)\\[4ex]
    \end{center}
\end{titlepage}

\newpage

\section*{L\"osungen zu Aufgabe 1}

\subsection*{Aufgabenteil a}

Der Quellcode ist in der Datei \href{../src/r1/Aufgabe1.gms}{Aufgabe1.gms} zu finden.

\subsection*{Aufgabenteil b}

Die entsprechende Ausf\"uhrung kann unter \href{../results/r1/Aufgabe1.lst}{Aufgabe1.lst} eingesehen werden. Daraus l\"asst sich entnehmen, dass der optimale Wert bei $5,169180$ Euro liegt und der optimale Punkt bei
\[
\begin{pmatrix}
\text{Apfel} \\ 
\text{Cornflakes} \\ 
\text{Karotten} \\ 
\text{Kartoffeln} \\ 
\text{K\"ase} \\ 
\text{Milch} \\ 
\text{Schokolade} \\ 
\text{Spinat} \\ 
\text{Steak}
\end{pmatrix}^T
=
\begin{pmatrix}
0 \\ 
3 \\ 
4 \\ 
0,4512 \\ 
0 \\ 
0 \\ 
2,0111 \\
0 \\
1
\end{pmatrix}^T
\]
zu finden ist.

\newpage

\section*{L\"osungen zu Aufgabe 2}

Die Unsicherheitsmenge l\"asst sich allgemein folgenderma\ss en beschreiben:

\[
\mathcal{U} = \left\{
\underbrace{
\left[
\begin{array}{c|c}
(c^0)^\top & d^0 \\ \hline
A^0 & b^0
\end{array}
\right]
}_{D^0}
+ \sum_{\ell=1}^L \zeta_\ell
\underbrace{
\left[
\begin{array}{c|c}
(c^\ell)^\top & d^\ell \\ \hline
A^\ell & b^\ell
\end{array}
\right]
}_{D^\ell}
\;\middle|\; \zeta \in \mathbb{Z}
\right\}
\]

F\"ur die Aufgabe werden unter anderem die Zielfunktion, gegeben durch
\[
    \min \quad 0.22x_1 + 0.18x_2 + 0.07x_3 + 0.14x_4 + 0.55x_5 + 0.1x_6 + 0.54x_7 + 0.28x_8 + 3.2x_9
\]
und die zweite Nebenbedingung 
\[
0.35x_1 + 7x_2 + x_3 + 2x_4 + 25x_5 + 3.5x_6 + 9x_7 + 2.5x_8 + 21x_9 \geq 56
\]
ben\"otigt.

Zuerst gilt es, die zweite Nebenbedingung in Standardform zu bringen. Dazu wird die Ungleichung mit $-1$ multipliziert.
\[
    -0.35x_1  -7x_2  -x_3  -2x_4 - 25x_5 - 3.5x_6 - 9x_7 - 2.5x_8 - 21x_9 \leq -56
\]

Aus den entsprechenden Vor\"uberlegungen l\"asst direkt die nominale Datenmatrix $D^0_2$ bilden:
\[
    D^0_2 = \begin{pmatrix}
        0.22 & 0.18 & 0.07 & 0.14 & 0.55 & 0.1 & 0.54 & 0.28 & 3.2 & 0 \\
        -0.35 & -7 & -1 & -2 & -25 & -3.5 & -9 & -2.5 & -21 & -56
    \end{pmatrix}
\]
        
Die Unsicherheiten betreffen s\"amtliche Koeffizienten der Zielfunktion, sowie die N\"ahrwerte f\"ur Proteine, Fette, Calcium und Vitamin B2. Zus\"atzlich gibt es eine Schwankung im Mindestbedarf f\"ur Proteine von $10g$. Alle Werte k\"onnen den Tabellen in der Aufgabenstellung entnommen werden.

Daher ist es entscheidend, die Unsicherheiten der Ziel- und Nebenfunktionen in absoluten Werten zu ermitteln. Die Unsicherheitsgr\"o\ss e kann anschlie\ss end wie folgt dargestellt werden:
\[
    \text{ZF} = \begin{pmatrix}
        0.06 & \underbrace{0.027}_{\mathrm{0.18 \times 0.15}} & \underbrace{0.014}_{\mathrm{0.07 \times 0.2}} & 0.04 & 0.1 & \underbrace{0.025}_{\mathrm{0.1 \times 0.25}} & \underbrace{0.216}_{\mathrm{0.54 \times 0.4}} & 0.1 & \underbrace{1.28}_{\mathrm{3.2 \times 0.4}}
    \end{pmatrix}        
\]
\[
    \text{NF} = \begin{pmatrix}
        \underbrace{-0.07}_{\mathrm{-0.35 \times 0.2}} & \underbrace{-0.7}_{\mathrm{-7 \times 0.1}} & \underbrace{-0.2}_{\mathrm{-1 \times 0.2}} & \underbrace{-0.1}_{\mathrm{-2 \times 0.05}} & \underbrace{-0.25}_{\mathrm{-25 \times 0.01}} & \underbrace{-0.35}_{\mathrm{-3.5 \times 0.1}} & \underbrace{-0.09}_{\mathrm{-9 \times 0.01}} & \underbrace{-0.25}_{\mathrm{-2.5 \times 0.1}} & \underbrace{-3.15}_{\mathrm{-21 \times 0.15}}
    \end{pmatrix}
\]

Ausgehend von diesen Werten lassen sich nun alle Shift-Matrizen $D^\ell$ berechnen. Die Anzahl der Unsicherheiten betr\"agt $L = 19$, woraus sich eine identische Anzahl an Shift-Matrizen ergibt. Diese sind wie folgt:

\[
D^1_2 = \begin{pmatrix}
    0.06 & 0 & 0 & 0 & 0 & 0 & 0 & 0 & 0 & 0 \\
    0 & 0 & 0 & 0 & 0 & 0 & 0 & 0 & 0 & 0
\end{pmatrix}
\]
\[
D^2_2 = \begin{pmatrix}
    0 & 0.027 & 0 & 0 & 0 & 0 & 0 & 0 & 0 & 0 \\
    0 & 0 & 0 & 0 & 0 & 0 & 0 & 0 & 0 & 0
\end{pmatrix}
\]
\[
D^3_2 = \begin{pmatrix}
    0 & 0 & 0.014 & 0 & 0 & 0 & 0 & 0 & 0 & 0 \\
    0 & 0 & 0 & 0 & 0 & 0 & 0 & 0 & 0 & 0
\end{pmatrix}
\]
\[
D^4_2 = \begin{pmatrix}
    0 & 0 & 0 & 0.04 & 0 & 0 & 0 & 0 & 0 & 0 \\
    0 & 0 & 0 & 0 & 0 & 0 & 0 & 0 & 0 & 0
\end{pmatrix}
\]
\[
D^5_2 = \begin{pmatrix}
    0 & 0 & 0 & 0 & 0.1 & 0 & 0 & 0 & 0 & 0 \\
    0 & 0 & 0 & 0 & 0 & 0 & 0 & 0 & 0 & 0
\end{pmatrix}
\]
\[
D^6_2 = \begin{pmatrix}
    0 & 0 & 0 & 0 & 0 & 0.025 & 0 & 0 & 0 & 0 \\
    0 & 0 & 0 & 0 & 0 & 0 & 0 & 0 & 0 & 0
\end{pmatrix}
\]
\[
D^7_2 = \begin{pmatrix}
    0 & 0 & 0 & 0 & 0 & 0 & 0.216 & 0 & 0 & 0 \\
    0 & 0 & 0 & 0 & 0 & 0 & 0 & 0 & 0 & 0
\end{pmatrix}
\]
\[
D^8_2 = \begin{pmatrix}
    0 & 0 & 0 & 0 & 0 & 0 & 0 & 0.1 & 0 & 0 \\
    0 & 0 & 0 & 0 & 0 & 0 & 0 & 0 & 0 & 0
\end{pmatrix}
\]
\[
D^9_2 = \begin{pmatrix}
    0 & 0 & 0 & 0 & 0 & 0 & 0 & 0 & 1.28 & 0 \\
    0 & 0 & 0 & 0 & 0 & 0 & 0 & 0 & 0 & 0
\end{pmatrix}
\]
\[
D^{10}_2 = \begin{pmatrix}
    0 & 0 & 0 & 0 & 0 & 0 & 0 & 0 & 0 & 0 \\
    0.07 & 0 & 0 & 0 & 0 & 0 & 0 & 0 & 0 & 0
\end{pmatrix}
\]
\[
D^{11}_2 = \begin{pmatrix}
    0 & 0 & 0 & 0 & 0 & 0 & 0 & 0 & 0 & 0 \\
    0 & 0.7 & 0 & 0 & 0 & 0 & 0 & 0 & 0 & 0
\end{pmatrix}
\]
\[
D^{12}_2 = \begin{pmatrix}
    0 & 0 & 0 & 0 & 0 & 0 & 0 & 0 & 0 & 0 \\
    0 & 0 & 0.2 & 0 & 0 & 0 & 0 & 0 & 0 & 0
\end{pmatrix}
\]
\[
D^{13}_2 = \begin{pmatrix}
    0 & 0 & 0 & 0 & 0 & 0 & 0 & 0 & 0 & 0 \\
    0 & 0 & 0 & 0.1 & 0 & 0 & 0 & 0 & 0 & 0
\end{pmatrix}
\]
\[
D^{14}_2 = \begin{pmatrix}
    0 & 0 & 0 & 0 & 0 & 0 & 0 & 0 & 0 & 0 \\
    0 & 0 & 0 & 0 & 0.25 & 0 & 0 & 0 & 0 & 0
\end{pmatrix}
\]
\[
D^{15}_2 = \begin{pmatrix}
    0 & 0 & 0 & 0 & 0 & 0 & 0 & 0 & 0 & 0 \\
    0 & 0 & 0 & 0 & 0 & 0.35 & 0 & 0 & 0 & 0
\end{pmatrix}
\]
\[
D^{16}_2 = \begin{pmatrix}
    0 & 0 & 0 & 0 & 0 & 0 & 0 & 0 & 0 & 0 \\
    0 & 0 & 0 & 0 & 0 & 0 & 0.09 & 0 & 0 & 0
\end{pmatrix}
\]
\[
D^{17}_2 = \begin{pmatrix}
    0 & 0 & 0 & 0 & 0 & 0 & 0 & 0 & 0 & 0 \\
    0 & 0 & 0 & 0 & 0 & 0 & 0 & 0.25 & 0 & 0
\end{pmatrix}
\]
\[
D^{18}_2 = \begin{pmatrix}
    0 & 0 & 0 & 0 & 0 & 0 & 0 & 0 & 0 & 0 \\
    0 & 0 & 0 & 0 & 0 & 0 & 0 & 0 & 3.15 & 0
\end{pmatrix}
\]
\[
D^{19}_2 = \begin{pmatrix}
    0 & 0 & 0 & 0 & 0 & 0 & 0 & 0 & 0 & 0 \\
    0 & 0 & 0 & 0 & 0 & 0 & 0 & 0 & 0 & 10
\end{pmatrix}
\]

\newpage

\end{document}
